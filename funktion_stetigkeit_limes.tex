\section{Stetigkeit und Limes einer Funktion}
$\lim_{x \to a} f(x) = b$ bedeutet, dass die Funktion $f$ für $x \to a$ den
Grenzwert $b$ hat. Der Funktionswert nähert sich also immer näher an $b$ heran,
wenn $x$ sich $a$ annähert (Epsilon-Delta-Kriterium).
\[
\forall \epsilon > 0 \; \exists \delta > 0: \|x -a\| < \delta \Rightarrow
\|f(x) - f(a)\| < \epsilon
\]

Funktionsfolgen verhalten sich desweiteren wie Folgen. Es gelten also die
gleichen Eigenschaften.

\subsection{(Punktweise) Stetigkeit}
Die Definition für punktweise Stetigkeit ist die gleiche, wie die für
$\lim_{x \to a} f(x) = b$ (siehe die Definition gerade über diesem Kapitel).

Charakterisierungen der Stetigkeit von $f$ im Punkt $a$:
\begin{itemize}
	\item $\lim_{x \to a} f(x) = f(a)$ ist die Definition
	\item Man kann $\lim_{x \to a} f(x)$ auch als $f(\lim_{x \to a} x)$ schreiben.
	Also die Reihenfolge zwischen Bildung des Grenzwertes und der Anwendung der
	Funktion vertauschen. Beispiel: $\lim_{x \to \infty} \sin\frac{1}{x} =
	\sin(\lim_{x \to \infty} \frac{1}{x}) = \sin(0) = 0$
	\item Es gilt linker Grenzwert = Funktionswert = rechter Grenzwert: $\lim_{x
	\nearrow a} f(x) = f(a) = \lim_{x \searrow a} f(x)$
	\item Für eine beliebige (jede) Folge $(x_n)$ mit $x_n \to a$ hat $f(x_n) \to
	f(a)$ zu gelten. Dies ist praktisch um zu zeigen, dass eine Funktion an einer
	Stelle nicht stetig ist.
\end{itemize}

Eine Funktion $f$ ist stetig, wenn sie in allen Punkten stetig ist
(``punktweise Stetigkeit'').

Es gelten folgenden Eigenschaften:
\begin{itemize}
	\item Ist $f$ und $g$ stetig in einem gemeinsamen Definitionsbereich, so sind
	$f + g, f- g, f \cdot g, \frac{f}{g}, f \circ g$ ebenfalls stetig.
\end{itemize}

\subsection{Gleichmässige Stetigkeit}
Sei $f: D \to \R$, $D \subseteq \R$, dann ist $f$ genau dann stetig wenn gilt:
\[
\forall \epsilon > 0 \exists \delta > 0 \forall x,x_0 \in D: |x - x_0| < \delta
\Rightarrow |f(x) - f(x_0)| < \epsilon
\]

Der Unterschied zur punktweisen Stetigkeit liegt darin, dass $\delta$ und
$\epsilon$ nicht auch noch von $x_0$ abhängig sind. So ist $f(x) = x^2$ zwar
(punktweise) stetig, aber nicht gleichmässig stetig. Begründung: Je weiter
rechts man zwei Punkte mit einem Abstand kleiner als $\delta$ wählt, desto
grösser wird der Abstand der beiden Funktionswerte. Dieser Abstand der
Funktionswerte müsste aber kleiner als das vorgegebene $\epsilon$ bleiben.

\subsection{Lipschitz-Stetigkeit}
Eine Funktion $f: \R \to \R$ ist Lipschitz-stetig, wenn eine Konstante $L$
existiert, so dass gilt:
\[
\forall x_1,x_2 \in \R: \|f(x_1) - f(x_2)\| \leq L \cdot \|x_1 - x_2\|
\]

\subsection{Stetigkeit, Differenzierbarkeit, Integrierbarkeit}
Sei $f$ eine Funktion, so gilt:

$f$ differenzierbar $\Rightarrow$ $f$ stetig $\Rightarrow$ $f$ integrierbar\\
\\
$f$ \emph{nicht} integrierbar $\Rightarrow$ $f$ \emph{nicht} stetig $\Rightarrow$ $f$\emph{nicht} differenzierbar  

\subsection{Abhängigkeit der Stetigkeitsbegriffe}
Sei $f$ eine reelle Funktion, so gilt:

$f$ Lipschitz-stetig $\Rightarrow$ $f$ absolut stetig $\Rightarrow$ $f$
gleichmässig stetig $\Rightarrow$ $f$ (punktweise) stetig.

\subsection{Regel von de l'Hospital}
Sei $a \in \R \cup \{\infty, -\infty\}$. Es hat zu gelten:
\begin{itemize}
  \item $\lim_{x \to a} f(x) = \lim_{x \to a} g(x) = 0 \text{ oder } \pm\infty$
  \item In der Nähe von $a$ ist $g'(x) \neq 0$
\end{itemize}

Dann ist
\[
\lim_{x \to a} \frac{f(x)}{g(x)} = \lim_{x \to a}
\frac{f'(x)}{g'(x)}
\]

Diese Regel kann man mehrfach anwenden hintereinander. \underline{Achtung:} Es
kann sein, dass der Limes $\lim_{x \to a} \frac{f'(x)}{g'(x)}$ nicht existiert,
aber von $\lim_{x \to a} \frac{f(x)}{g(x)}$ trotzdem.
