\section{Differentialgleichung (DGL)}
\subsection{Lineare DGL 1. Ordnung ($y' + f(x) \cdot y = g(x)$)}
Wenn $g(x) = 0$ ist, dann ist die DGL homogen. Falls $g(x) \neq 0$, so handelt
es sich um eine inhomogene DGL.

Der erste Schritt für homogene und inhomogene DGL ist die Lösung der homogenen
DGL: $y' + f(x) \cdot y = 0$:
{\small
\begin{align*}
y' + f(x) \cdot y &= 0 \quad \left | -(f(x) \cdot x) \right.\\
y' &= -f(x) \cdot y \quad \boxed{y' \text{ ist das gleiche wie } \frac{dy}{dx}}\\
\frac{dy}{dx} &= -f(x) \cdot y \quad \left | \div y \right.\\
\frac{dy}{dx\, y} &= -f(x) \quad \left | \int \right.\\
\int \frac{dy}{dx\, y} dx &= \int -f(x) dx \quad \boxed{\frac{dy}{dx\, y} \cdot dx = \frac{dx}{dx\, y} dy = \frac{1}{y} dy}\\
\int \frac{1}{y} dy &= \int -f(x) dx\\
\ln(y) &= -F(x) \quad \left | e^\alpha \right.\\
e^{\ln(y)} &= e^{-F(x)}\\
y &= e^{-F(x)}
\end{align*}
}

Damit erhalten wir die allgemeine Lösung: $y = A \cdot e^{-F(x)}$. Hat man eine
homogene DGL und einen Punkt, an dem die ursprüngliche Funktion ausgewertet wurde,
so kann man die explizite Lösung berechnen (also $A$ berechnen), in dem man die
hier allgemein erhaltene Lösung für den gegebenen Punkt auswertet und so die
Unbekannte bekommt.

Für ein inhomogenes DGL setzt sich die allgemeine Lösung aus der homogenen Lösung
$y_h$ und der partikulären (speziellen) Lösung $y_p$ der inhomogenen DGL zusammen.
Die homogene Lösung haben wir bereits berechnet: $y_h = A \cdot e^{-F(x)}$. Nun
folgt die partikuläre Lösung:

Dazu wird die Konstante ($A$) der homogenen Lösung als Funktion dargestellt ($u(x)$).
Wir erhalten somit: $y_p = u(x) \cdot e^{-F(x)}$.
Dieses $y_p$ setzten wir nun als $y$ in die inhomogene Gleichung ein:
{\small
\[
y' + f(x) \cdot y = g(x) \Rightarrow (\underbrace{u(x) \cdot e^{-F(x)}}_{= y_p = y})'
+ f(x) \cdot (\underbrace{u(x) \cdot e^{-F(x)}}_{= y_p = y}) = g(x)
\]
}

Die neue Gleichung wird nun nach $u'(x) = \ldots$ aufgelöst, was zu
$u'(x) = \frac{g(x)}{e^{-F(x)}}$ führt. Nun wird $u(x)$ bestimmt durch integrieren
beider Seiten: $u(x) = \int \frac{g(x)}{e^{-F(x)}}\,dx$. Hat man dies ausgerechnet,
setzt man $u(x)$ in $y_p = u(x) \cdot e^{-F(x)}$ ein und bekommt so die partikuläre
Lösung der DGL.

Als letzter Schritt für inhomogene DGL summiert man $y_h$ und $y_p$ und erhält nach
dem Umformen und Kürzen die allgemeine Lösung der DGL:
{\small
\[
y = y_h + y_p = 
\underbrace{A \cdot e^{-F(x)}}_{= y_h} +
\underbrace{\underbrace{\int \frac{g(x)}{e^{-F(x)}}\,dx}_{= u(x)} \cdot e^{-F(x)}}_{= y_p}
\]
}

Hat man für die inhomogene DGL ebenfalls Punkte an denen die Funktion ausgewertet wurde,
so kann man dies in die allgemeine Lösung eintragen und so die Unbekannten ($A$) berechnen.

\subsubsection{Beispiel}
Gegeben: $y' + x^2 \cdot y = 2x^2$

Homogene DGL lösen: $y' + x^2 \cdot y = 0$
\begin{align*}
y' + x^2 \cdot y &= 0\\
\frac{dy}{dx} + x^2 \cdot y &= 0 \quad | -(x^2 \cdot y)\\
\frac{1}{dx}\, dy &= -x^2 \cdot y \quad | \div y\\
\frac{1}{dx} \frac{1}{y} \, dy &= -x^2 \quad | \int\\
\int \frac{1}{dx} \frac{1}{y} \, dy \, dx &= \int -x^2 \, dx\\
\int \frac{1}{y}\, dy &= \int -x^2 \, dx\\
\ln(y) &= -\frac{1}{3} x^3 \quad | e^\alpha\\
y &= e^{-\frac{1}{3}x^3}
\end{align*}

Somit ist die allgemeine homogene Lösung: $y_h = A \cdot e^{-\frac{1}{3}x^3}$.


Als nächstes gehen wir die praktikuläre Lösung an:
$y_p = u(x) \cdot e^{-\frac{1}{3}x^3}$
\begin{align*}
\Rightarrow (u(x) \cdot e^{-\frac{1}{3}x^3})' + x^2 (u(x) e^{-\frac{1}{3}x^3}) &= 2 x^2\\
u'(x) \cdot e^{-\frac{1}{3}x^3} - u(x) \cdot x^2 e^{-\frac{1}{3}x^3} + u(x) x^2 e^{-\frac{1}{3}x^3} &= 2 x^2\\
u'(x) \cdot e^{-\frac{1}{3}x^3} &= 2 x^2 \quad | \div e^{-\frac{1}{3}x^3}\\
u'(x) &= 2 x^2 e^{\frac{1}{3}x^3} \quad | \int\\
u(x) &= 2 e^{\frac{1}{3}x^3}
\end{align*}

Wir erhalten somit: $y_p = 2 e^{\frac{1}{3}x^3} \cdot e^{-\frac{1}{3}x^3} = 2$.
Die allgemeine Lösung des inhomogenen DGL ist somit:
$\underline{\underline{y}} = y_h + y_p = \underline{\underline{A \cdot e^{-\frac{1}{3}x^3} + 2}}$