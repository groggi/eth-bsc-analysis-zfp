\section{Funktionsfolgen}
$\lim_{x \to a} f(x) = b$ bedeutet, dass die Funktion $f$ für $x \to a$ den
Grenzwert $b$ hat. Der Funktionswert nähert sich also immer näher an $b$ heran,
wenn $x$ sich $a$ annähert.
\[
\forall \epsilon > 0 \; \exists \delta > 0: 0 < |x -a| < \delta \Rightarrow
|f(x) - f(a)| < \epsilon
\]

Funktionsfolgen verhalten sich desweiteren wie Folgen. Es gelten also die
gleichen Eigenschaften.

\subsection{Regel von de l'Hospital}
Sei $a \in \R \cup \{\infty, -\infty\}$. Es hat zu gelten:
\begin{itemize}
  \item $\lim_{x \to a} f(x) = \lim_{x \to a} g(x) = 0 \text{ oder } \pm\infty$
  \item In der Nähe von $a$ ist $g'(x) \neq 0$
\end{itemize}

Dann ist
\[
\lim_{x \to a} \frac{f(x)}{g(x)} = \lim_{x \to a}
\frac{f'(x)}{g'(x)}
\]

Diese Regel kann man mehrfach anwenden hintereinander. \underline{Achtung:} Es
kann sein, dass der Limes $\lim_{x \to a} \frac{f'(x)}{g'(x)}$ nicht existiert,
aber von $\lim_{x \to a} \frac{f(x)}{g(x)}$ trotzdem.

\subsection{Stetigkeit}
Charakterisierungen der Stetigkeit von $f$ im Punkt $a$:
\begin{itemize}
	\item $\lim_{x \to a} f(x) = f(a)$ ist die Definition
	\item Man kann $\lim_{x \to a} f(x)$ auch als $f(\lim_{x \to a} x)$ schreiben.
	Also die Reihenfolge zwischen Bildung des Grenzwertes und der Anwendung der
	Funktion vertauschen. Beispiel: $\lim_{x \to \infty} \sin\frac{1}{x} =
	\sin(\lim_{x \to \infty} \frac{1}{x}) = \sin(0) = 0$
	\item Es gilt linker Grenzwert = Funktionswert = rechter Grenzwert: $\lim_{x
	\nearrow a} f(x) = f(a) = \lim_{x \searrow a} f(x)$
	\item Für eine beliebige (jede) Folge $(x_n)$ mit $x_n \to a$ hat $f(x_n) \to
	f(a)$ zu gelten. Dies ist praktisch um zu zeigen, dass eine Funktion an einer
	Stelle nicht stetig ist.
\end{itemize}