\section{Taylorreihe / -entwicklung}
Funktionen werden in der Umgebung eines bestimmten Punktes durch eine
Potenzreihe dargestellt. Damit wird die ursprüngliche Funktion angenähert.

Die Taylorreihe der Funktion $f$ mit Grad $k$ um den Punkt $a$ ist:
{\small
\[
P_a^f(x) = \sum_{n = 0}^k \frac{f^{(n)}(a)}{n!}(x - a)^n = f(a) +
\frac{f'(a)}{1!}(x-a) + \frac{f''(a)}{2!}(x-a)^2 + \ldots
\]
}

\todo{Wie Taylorreihe aus den einzelnen Taylorreihen erstellen bei Komposition
/ Multiplikation der Funktionen}

\subsection{Rechenregeln}
Sind $f,g$ beliebig oft differenzierbar, so gilt
\begin{itemize}
  \item $T^{f+g}(x) = T^f + T^g$
  \item $T^{f \cdot g}(x) = T^f \cdot T^g$
\end{itemize}

\subsubsection{Kettenregel}
Sei $f: A \to B, g: B \to C$ und wir möchten die Taylorentwicklung von $g \circ
f = f(g(x))$ berechnen. Sei $a \in A$ unser Punkt um den wir Entwickeln. Nun
müssen wir $s = f(a)$ berechnen. Dann gilt:
{\small
\[
T^{f(g)}_a (x) = T^f_a(T^g_s)(x)
\]
}

Man setzt als das Resultat $s = f(a)$ als Entwicklungspunkt für die Taylorreihe
von $g$.

\todo{Prüfen ob das so stimmt!}
