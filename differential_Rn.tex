\section{Differentialrechnung in $\R^n$}
Hier geht es um Funktionen $f: \R^n \to \R^m$, wobei $m=1$ gelten kann
($f: \R^n \to \R$). Solche Funktionen haben die allgemeine Form:
$f(x) = f(x_1, x_2, x_3, \ldots, x_n) = \begin{pmatrix}
f_1(x_1, x_2, x_3, \ldots, x_n)\\
f_2(x_1, x_2, x_3, \ldots, x_n)\\
\ldots\\
f_m(x_1, x_2, x_3, \ldots, x_n)
\end{pmatrix}$

Für nahezu alle Eigenschaften gilt: Die Vektorfunktion $f: \R^n \to \R^m$ hat
eine bestimmte Eigenschaft, wenn jede einzelne ihrer Komponenten
($f_1, f_2, \ldots, f_m$) die besagte Eigenschaft besitzen. Das Problem liegt
neu also nicht im Wertebereich, sondern vor allem in Definitionsbereich.

\subsection{Norm}
Eine Norm auf $\R^n$ ist die Funktion $\|\cdot\|: \R^n \to \R$ mit den folgenden
Eigenschaften:
\begin{itemize}
	\item $\forall x \in \R^n: \|x\| \geq 0$
	\item $\forall x \in \R^n: \|x\| = 0 \Leftrightarrow x = \vec{0}$
	\item $\forall x \in \R^n, \alpha \in \R: \|\alpha x\| = |\alpha| \|x\|$
	\item $\forall x,y \in \R^n: \|x + y\| \leq \|x\|+\|y\|$
\end{itemize}

\subsection{Partielle Differenzierbarkeit}
$f: \R^n \to \R^m$ ist in $a = (a_1, \ldots, a_n)$ partiell differenzierbar nach
der $i$-ten Variable $x_i$, wenn die Funktion
$f: x_i \to f(x_1, \ldots, x_i, \ldots, x_n)$ differnzierbar ist. Man berechnet
die partielle Ableitung also folgendermassen: Eine Funktion $f$ wird nach einer
Variable partiell differenziert, indem man alle anderen Variablen als Konstanten
behandelt und die Rechenregeln für Funktionen mit einer Variable anwendet.

\begin{satz}[Satz von Schwarz]
Ist $f$ nach $x$ und $y$ zweimal partiell differenzierbar und sind die gemischten
partiellen Ableitungen $f_{xy}$ und $f_{yx}$ stetig, so gilt: $f_{xy} = f_{yx}$.
\end{satz}