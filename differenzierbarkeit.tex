\section{Differenzierbarkeit}
\subsection{Definition}
$f$ ist in $a \in I$ differenzierbar mit der Ableitung $f'(a)$, wenn
\[
\lim_{x \to a} \frac{f(x) - f(a)}{x - a} =: f'(a) = \frac{d}{dx}f(a)
\]
existiert.

Ist $f'$ stetig im Definitionsbereich, so heisst $f$ stetig differenzierbar. Man
kann also $f$ differenzieren und bekommt mit $f'$ eine stetige Funktion. Es gilt
auch $f \in C^1(I)$. $C^n(I)$ ist die Menge der $n$-mal stetig differenzierbaren
Funktionen über dem Intervall $I$.

\subsection{Mittelwertsatz (Satz von Lagrange)}
Ist $f$ au $[a,b]$ stetig und in $]a, b[$ differenzierbar, so gibt es ein $c
\in ]a,b[$ mit
\[
\frac{f(b) - f(a)}{b-a} = f'(c)
\]

\subsubsection{Beispiel: Ungleichungen}
Zu zeigen: $e^x(y-x) < e^y - e^x < e^y(y-x), \; \forall x < y$:


Der Mittelwertsatz wird auf die Exponentialfunktion angewendet. Damit gilt für
ein Paar von Zahlen $x < y$ ein $u \in ]x,y[$, für welches gilt: $\frac{e^y
- e^x}{y-x} = e^u$. Weil die Exponentialfunktion in $\R$ streng monoton wachsend
ist, gilt $e^x < e^u < e^y$ und somit gilt: $e^x < \frac{e^y-e^x}{y-x} < e^y$.
Multipliziert man nun mit dem Nenner des Bruchs bekommt man: $e^x (y-x) < e^y -
e^x < e^y(y-x)$.

\subsection{Monotonie}
\begin{itemize}
	\item $f' > 0 \Rightarrow f$ streng monoton steigend
	\item $f' \geq 0 \Leftrightarrow f$ monoton steigend
	\item $f' < 0 \Rightarrow f$ streng monoton fallend
	\item $f' \leq 0 \Leftrightarrow f$ monoton fallend
\end{itemize}
Insbesondere besitzt die Funktion $f$ wenn sie streng monoton wachsend/fallend ist
weder kritische Punkte, noch lokale oder globale Extrema.

\subsection{Konvexität}
\begin{definition}[konvex]
Eine Funktion $f$ ist konvex, wenn $\forall a,b: f(\frac{a + b}{2}) \leq \frac{f(a) + f(b)}{2}$
\end{definition}

\begin{definition}[konkav]
Eine Funktion $f$ ist konkav, wenn $\forall a,b: f(\frac{a + b}{2}) \geq \frac{f(a) + f(b)}{2}$
\end{definition}

\underline{Anschaulich:} Anschaulich bedeutet das, dass bei einer konvexen Funktion
der Graph immer unter und bei einer konkaven stets über der Sekante liegt. Der Graph
konvexer Funktionen ist linksgekrümmt, der konkaver Funktionen ist rechtsgekrümmt.

\begin{itemize}
	\item $f'' \geq 0 \Leftrightarrow f$ konvex
	\item $f'' \leq 0 \Leftrightarrow f$ konkav
\end{itemize}

\subsubsection{Extremstellen}
\begin{itemize}
	\item $f'(x_0) = 0, f''(x_0) > 0 \Rightarrow$ Minimum bei $x_0$
	\item $f'(x_0) = 0, f''(x_0) < 0 \Rightarrow$ Maximum bei $x_0$
	\item $f''(x_0) = 0, f'''(x_0) \neq 0 \Rightarrow$ Wendepunkt in $x_0$
	\item $f'(x_0) = 0, f''(x_0) = 0, f'''(x_0) \neq 0 \Rightarrow$ Sattelpunkt in
  	$x_0$
	\item Extrema bei $x_0 \Rightarrow f'(x_0) = 0$
\end{itemize}

\subsubsection{Beispiel Wendepunkt berechnen}
%Von http://www.frustfrei-lernen.de/mathematik/wendepunkt-berechnen.html
Die hinreichende Bedingung für einen Wendepunkt lautet: $f''(x_0) = 0$ und $f'''(x_0) \not= 0$\\
 
Praktische Vorgehensweise für $f(x) = \frac{1}{3}x^3-2x^2+3x$:\\
Um eine Funktion auf Wendepunkte hin zu untersuchen, führen wir die folgenden Schritte durch:
\begin{itemize}
	\item Wir leiten die Funktion f(x) dreimal ab. $f'(x) = x^2-4x+3$, $f''(x) = 2x-4$ und $f'''(x) = 2$
	\item Wir setzen die zweite Ableitung Null und berechnen den X-Wert, sofern möglich  
		$f''(x) = 2x-4 \stackrel{!}{=} 0 \Rightarrow x = 2$ 
	\item Sofern möglich, setzen wir diesen X-Wert in die dritte Ableitung ein
		$f'''(x) = 2$
	\item Ist dieses Ergebnis ungleich Null, liegt ein Wendepunkt vor
		$f'''(x) = 2 \not= 0 \Rightarrow$ Wendepunkt
	\item Der X-Wert wird in f(x) eingesetzt, um den zugehörigen Y-Wert zu bestimmen.
		$x = 2$ in $f(x)$ einsetzen: $f(2) = \frac{1}{3}2^3-2\cdot2^2+3\cdot2 = \frac{2}{3}$
\end{itemize}

\subsection{Zusammenhang zwischen Stetigkeit und Differenzierbarkeit}
\begin{itemize}
	\item differenzierbar $\Rightarrow$ stetig
	\item differenzierbar $\not\Leftarrow$ stetig
\end{itemize}

\subsection{Umkehrsatz}
% aus http://mathematik-netz.de/pdf/Umkehrsatz.pdf
Ist $f:I \to \R$ auf I differenzierbar mit $f'(x) \neq 0$ für jedes $x \in I$,
so ist $f$ streng monoton wachsend oder streng monoton fallend und damit
injektiv. Die Umkehrfunktion $f^{-1}:f(I) \to I$ existiert und ist ebenfalls
streng monoton wachsend oder streng monoton fallend. Die Ableitung
$(f^{-1})'(y)$ existiert für alle $y \in f(I)$ mit $f'(f^{-1}(y)) \neq 0)$, und zwar gilt dann \[ (f^{-1})'(y) = \frac{1}{f'(f^{-1}(y))}
\]

\subsection{Monotonie, Bijektion, Differenzierbarkeit}
\begin{satz}
(\textit{Folgendes gilt auch für streng monoton fallende Funktionen})

Sei $f: [a, b] \to \R$ stetig und streng monoton wachsend. Es sei dann
$A = f(a), b = f(b)$, dann gilt: $f: [a,b] \to [A, B]$ ist bijektiv.

$f$ hat also eine Umkehrfunktion $f^{-1}: [A,B] \to [a,b]$ und diese ist stetig
und streng monoton wachsend.
\end{satz}

\underline{Tipp:} Differentierbare Funktionen sind stetig.
