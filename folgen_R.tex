\section{Folgen in $\R$}

\subsection{Definitionen}
\begin{description}
  \item[konvergent] $\lim_{x \to \infty} a_n$ existiert \index{konvergent}
  \item[divergent] $\lim_{x \to \infty} a_n$ existiert nicht \index{divergent}
  \item[Nullfolge] $\lim_{x \to \infty} a_n = 0$ gilt \index{Nullfolge}
  \item[beschränkt] Es gibt $C_1, C_2$, so dass gilt $C_1 \leq a_n \leq C_2$
  bzw. $C$ gibt, so dass $|a_n| \leq C$
  \item[unbeschränkt] falls $(a_n)$ nicht beschränkt ist. Unbeschränkte folgen
  sind stets \underline{divergent}
  \item[monoton wachsend] $a_n \leq a_{n+1} \quad \forall n \in \N$ \index{monoton}
  \item[streng monoton wachsend] $a_n < a_{n+1} \quad \forall n \in \N$
  \item[monoton fallend] $a_n \geq a_{n+1} \quad \forall n \in \N$
  \item[streng monoton fallend] $a_n > a_{n+1} \quad \forall n \in \N$
  \item[alternierend] die Vorzeichen der Folgenglieder wechseln sich ab
  \item[bestimmt divergent / uneigentlich konvergent] es gilt $\lim_{n \to
  \infty} a_n = \pm \infty$
\end{description}

\begin{definition}[Grenzwert] \index{Grenzwert}
	\begin{align*}
		\lim_{n \to \infty} a_n = a & \Leftrightarrow a_n \to a \\
		& \Leftrightarrow \forall \epsilon > 0 \exists n_0 \in \N \forall n \geq n_0:
		|a_n - a| < \epsilon
	\end{align*}
\end{definition}

\begin{definition}[Teilfolge] \index{Teilfolge}
Werden von einer Folge beliebig viele Glieder weggelassen, aber nur so viele,
dass noch unendlich viele übrigbleiben, so erhält man eine Teilfolge.
\end{definition}

\begin{definition}[Häufungspunkt] \index{Häufungspunkt}
	$a$ ist Häufungspunkt der Folge $(a_n)$, wenn in jeder Umgebung von $a$
	unendlich viele Folgeglieder liegen. Das ist äquivalent damit, dass $a$ der
	Limes einer Teilfolge von $(a_n)$ ist.
\end{definition}

\begin{definition}[Limes superior / Limes inferior] \index{$\limsup$}\index{$\liminf$}
	Ist $(a_n)$ eine beschränkte Folge, so heisst der grösste Häufungspunkt Limes
	superior ($\limsup_{n \to \infty} a_n$ oder $\overline{\lim}_{n \to \infty}
	a_n$). Der kleinste Häufungspunkt ist der Limes inferior ($\liminf_{n \to
	\infty} a_n$ oder $\underline{\lim}_{n \to \infty} a_n$)
\end{definition}

\subsection{Rechnen mit Eigenschaften}
Addition:
\begin{itemize}
  \item $(a_n), (b_n)$ konvergiert $\Rightarrow (a_n + b_n)$ konvergiert
  \item $(a_n)$ konvergiert, $(b_n)$ divergent $\Rightarrow (a_n + b_n)$
  divergent
  \item $(a_n)$ beschränkt, $(b_n)$ beschränkt $\Rightarrow (a_n + b_n)$
  beschränkt
  \item $(a_n)$ beschränkt, $(b_n)$ unbeschränkt $\Rightarrow (a_n + b_n)$
  unbeschränkt
  \item $(a_n)$ beschränkt, $(b_n) \to \pm \infty \Rightarrow (a_n + b_n) \to
  \pm \infty$
  \item $(a_n) \to \infty$, $(b_n) \to \infty \Rightarrow (a_n + b_n) \to \infty$
  \item $(a_n) \to -\infty$, $(b_n) \to -\infty \Rightarrow (a_n + b_n) \to
  -\infty$
\end{itemize}

Produkt:
\begin{itemize}
  \item $(a_n)$ Nullfolge, $(b_n)$ beschränkt $\Rightarrow (a_n b_n)$ Nullfolge
  \item $(a_n)$ konvergent, $(b_n)$ beschränkt $\Rightarrow (a_n b_n)$
  beschränkt
  \item $(a_n)$ konvergent, $(b_n)$ konvergent $\Rightarrow (a_n b_n)$
  konvergent
  \item $(a_n)$ konvergent gegen $a \neq 0$, $(b_n)$ divergent $\Rightarrow
  (a_n b_n)$ divergent
\end{itemize}

\subsection{Rechnen mit Grenzwerten}
$\lim_{n \to \infty} a_n = a$, $\lim_{n \to \infty} b_n = b$\\
\emph{\underline{Achtung!} Untenstehendes gilt \underline{nur} wenn die Grenzwerte von $a_n$ und $b_n$ existieren. (Nicht $0$ oder $\inf$ sind.)}
\begin{itemize}
  \item $\lim_{n \to \infty} (a_n \pm b_n) = a \pm b$
  \item $\lim_{n \to \infty} (c \cdot a_n) = c \cdot a$
  \item $\lim_{n \to \infty} (a_n b_n) = ab$
  \item \underline{Achtung:} $\lim_{n \to \infty} (a_n)^c = (\lim_{n \to
  \infty} a_n)^c$, nur wenn $c \neq n$
  \item $\lim_{n \to \infty} \frac{a_n}{b_n} = \frac{a}{b}, \quad (b_n)$ keine
  Nullfolge
\end{itemize}

\subsection{Hilfsmittel}
\textbf{Bernoullische Ungleichung}: Für $x \geq -1$ und $n \in \N$
\[
	(1+x)^n \geq 1 + nx
\]


\textbf{Vergleich von Folgen}: weiter rechts stehende Werte gehen schneller nach
$\infty$
\[
	1, \quad \ln n, \quad n^\alpha \; (\alpha > 0), \quad q^n \; (q > 1), \quad n!,
	\quad n^n
\]

\textbf{Stirlingformel}:
\[
	n! \approx \sqrt{2 \pi n} \left (\frac{n}{e} \right )^n
	\Rightarrow \left ( \frac{n}{e} \right )^n \sqrt{2 \pi n} \leq n! \leq \left (
	\frac{n}{e} \right )^n \sqrt{2 \pi n} \cdot e^\frac{12}{n}
\]

\subsection{Konvergenzkriterien}\index{Konvergenzkriterien}
\begin{align*}
	a_n \to a \Leftrightarrow a_n - a \to 0 \Leftrightarrow |a_n - a| \to 0
\end{align*}
	
\begin{itemize}[leftmargin=*]	
	\item Ist $\lim_{n \to \infty} a_n = a$, so ist der Limes $a$ einziger
	Häufungspunkt der Folge $(a_n)$ und jede Teilfolge konvergiert auch gegen $a$.
	
	\textbf{Beispiel:} Wegen $\left( 1 + \frac{1}{n} \right)^n \to e$, so gilt auch
	$\left( 1 + \frac{1}{2n} \right)^{2n} \to e$
	
	\item Hat die Folge zwei verschiedene Häufungspunkte, so ist die Folge sicher
	divergent.
	
	\item Ist die Folge monoton steigend und nach oben beschränkt, dann existiert
	$\lim_{n \to \infty} a_n$. Ist die Folge monoton fallend und nach unten
	beschränkt, dann existiert $\lim_{n \to \infty} a_n$
	
	\item Konvergiert $\sum_{n=0}^\infty a_n$, so ist $\lim_{n \to \infty} a_n =
	0$\\
	Damit kann die Regeln für Reihen verwenden. Siehe Grenzwerte von Reihen.
	
	\item Gibt es eine Funktion $f$ mit $f(n) = a_n$ und $\lim_{x \to \infty} f(x)
	= a$, so gilt auch $\lim_{n \to \infty} a_n = a$.\\
	Damit kann man zum Beispiel die Regel von \underline{l'Hospital} und die
	restlichen Methoden anwenden. Siehe Grenzwerte von Funktionen.\\
	\underline{Achtung:} Es kann sein, dass $f$ keinen Grenzwert besitzt, aber
	$(a_n)$ schon.
	
	\item \textbf{Einschliessungskriterium}: Sind $(a_n), (b_n), (c_n)$ Folgen mit
	$a_n \leq b_n \leq c_n$ und haben $(a_n), (c_n)$ den gleichen Grenzwert $a$, so
	konvergiert auch $(b_n)$ nach $a$.
\end{itemize}

\subsection{Tipps \& Beispiele}
\subsubsection{Brüche}
$\lim_{n \to \infty} \frac{n^2 + \ln n}{\sqrt{n^4 - n^3}}$

Bei Brüchen empfiehlt es sich den am stärksten wachsenden Teil (das am
schnellsten wachsende $n$) zu kürzen. In diesem Fall ist es das $n^4$ in der
Wurzel, also $n^2$.

\begin{align*}
\ldots &= \lim_{n \to \infty} \frac{n^2 + \ln n}{\sqrt{n^4 - n^3}}
\cdot \frac{\frac{1}{n^2}}{\frac{1}{n^2}} = \lim_{n \to \infty} \frac{n^2 + \ln
n}{n^2 \sqrt{1 - \frac{1}{n}}} \cdot \frac{\frac{1}{n^2}}{\frac{1}{n^2}} \\
&= \lim_{n \to \infty} \frac{1 + \frac{\ln n}{n^2}}{\sqrt{1 - \frac{1}{n}}}
= \frac{1 + 0}{\sqrt{1 - 0}} = 0
\end{align*}

\subsubsection{l'Hospital für Folgen (Folge als Funktion)}
$\lim_{n \to \infty} \frac{\ln n}{n^2}$

Die Funktion $f(x) = \frac{\ln x}{x^2}$ entspricht unseren Folgegliedern ($f(n)
= a_n = \frac{\ln n}{n^2}$). Für $n \to \infty$ hat der Nenner und der Zähler
den Grenzwert $\infty$, also wenden wir die Regel von l'Hospital an.

\begin{align*}
\ldots &= \lim_{x \to \infty} \frac{(\ln x)'}{(x^2)'} = \lim_{x \to \infty}
\frac{\frac{1}{x}}{2x} = \lim_{x \to \infty} \frac{1}{x^2} = 0
\end{align*}

Somit geht auch die Folge gegen 0.

\subsubsection{Wurzeln}
$\lim_{n \to \infty} (\sqrt{n^2 + an + 1} - \sqrt{n^2 + 1})$

Hier wendet man die dritte binomische Formel an, um den grenzwert zu berechnen.
Die einzelnen Terme streben jeweils gegen $\infty$ und $\infty - \infty$ kann
nicht berechnet werden.

\underline{Achtung} auf die Vorzeichen beim Anwenden der Regel!

\begin{align*}
&= \lim_{n \to \infty} (\sqrt{n^2 + an + 1} - \sqrt{n^2 + 1}) \cdot
\left(\frac{\sqrt{n^2 + an + 1} + \sqrt{n^2 + 1}}{\sqrt{n^2 + an + 1} +
\sqrt{n^2 + 1}} \right) \\
&= \lim_{n \to \infty} \frac{(n^2 + an + 1) - (n^2 + 1)}{\sqrt{n^2 + an + 1} +
\sqrt{n^2 + 1}} \\
&= \lim_{n \to \infty} \frac{an}{\sqrt{n^2 + an + 1} + \sqrt{n^2 + 1}} 
\end{align*}

nun verwenden wir den Tipp für Brüche und kürzen das $n$ heraus

\begin{align*}
\ldots &= \lim_{n \to \infty} \frac{a}{\sqrt{1 + \frac{a}{n} + \frac{1}{n^2}} +
\sqrt{1 + \frac{1}{n^2}}} = \frac{a}{1 + 1} = \frac{a}{2}
\end{align*}

\todo{Gelbes Rechenbuch 1, Seiten 163+ mit aufnehmen? Rekursive Folgen und so\ldots}

\subsubsection{Laufvariable im Exponent}
$\lim_{x \to 0} (3 - |x|)^{\frac{\sin(x)}{x}}$\newline
$\Rightarrow (3 - |x|)^{\frac{\sin(x)}{x}} = e^{\frac{\sin(x)}{x} \cdot \ln(3 -
|x|)}$\newline
$\Rightarrow 
\lim_{x \to 0} (3 - |x|)^{\frac{\sin(x)}{x}} = \lim_{x \to 0}e^{\frac{\sin(x)}{x} \cdot \ln(3 -
|x|)}$\newline
$\Rightarrow \lim_{x \to 0}\frac{\sin(x)}{x}\cdot \ln(3-|x|) = \ln(3)$\newline
$\Rightarrow \lim_{x \to 0}(3 - |x|)^{\frac{\sin(x)}{x}} = e^{\ln(3)} = 3$

\subsection{Cauchy-Folgen}\index{Cauchy-Folge}
\begin{definition}[Cauchy-Folge]
Sei $(a_n)_{n \in \N}$ eine Folge in $\R$. $(a_n)_{n \in \N}$ heisst \textbf{Cauchy-Folge}, falls gilt
\begin{align*}
\forall \epsilon > 0 \; \exists n_n = n_0(\epsilon) \in \N \; \forall n, l \geq n_0: |a_n - a_l| < \epsilon
\end{align*}
\end{definition}

Die Definition sagt grundsätzlich aus, dass ab einem $n_0(\epsilon)$ (also einem Anfang $n_0$, der abhängig von $\epsilon$ ist)
die Folgeglieder nur noch $\epsilon$ Abstand zu einander haben. Also der Abstand beliebig klein wird zwischen Folgegliedern.

\begin{satz}[Cauchy-Kriterium]
Für $(a_n)_{n \in \N} \subset \R$ sind äquivalent:
\begin{itemize}
	\item $(a_n)_{n \in \N}$ ist konvergent
	\item $(a_n)_{n \in \N}$ ist Cauchy-Folge
\end{itemize}
\end{satz}
